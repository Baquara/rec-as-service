\xchapter{Introdu\c{c}\~{a}o}{Este eh o primeiro cap\'{\i}tulo, onde eu conto toda a historia deste trabalho, o problema, a solu\c{c}\~{a}o, etc.}

% É recomendável utilizar `\acresetall' no início de cada capítulo para reiníciar o contator de referências às siglas.
\acresetall 

\section{Se\c{c}\~{a}o}
Trabalho do  \ac{PGCOMP}. Bolsa do \ac{CNPq}.

\begin{figure}[h]
    Figure
    \caption{As siglas também funcionam nas legendas, seja na forma de sigla \ac{CNPq}, seja na forma completa \acf{PGCOMP}.}
\end{figure}

\lipsum

\subsection{Uma Subse\c{c}\~{a}o}
\acresetall
Texto para mostrar como o \verb|\acresetall| funciona \ac{CNPq}, \ac{PGCOMP}. Ele reseta os contadoes e faz a sigla aparecer na forma estendida novamente.

\subsection{Outra Subse\c{c}\~{a}o}

Texto  \acf{CNPq}, \acf{PGCOMP}.