\xchapter{Web Architecture}
\label{chap:web_architecture}
\acresetall 

\section{Definition}

This section intends to provide a comprehensive explanation on the proposed software's Web Architecture. Since the application is being deployed on the Internet, it should follow a couple of its standards and architectures, in order to enable communication between different peers and allow interoperability between different systems; It should be able to use the default data transmission protocols, such as TCP/IP, HTTP and HTTPS, and it can be displayed on representation formats (HTML, CSS, XML) as long as the API is in place, and everything being accessible through the URL addressing standard. The application will use a Three-tier model of architecture: it have client-server inseparability, but with an extra layer of logic between them as it will be required Javascript and JSON objects which will be sent to API routes using the REST architecture. 

\section{Distribuited Web Service: SOA}
\acresetall

As there are complexities involved on the client-server model, the software needs to be designed around the idea of distributed web services particularly SOA (Service Oriented Architecture), since this would enable modularity and splitting the complete environment into functional units. Part of the role of SOA is developing a flexible and scalable application by utilizing loose coupling, heterogeneity and decentralized, where interoperability is a key feature. SOA also heavily relies on Representational state transfer (REST) architecture, which is the basis of this work. To summarize, SOA is a style of organizing (services), and Web services such as SaaS services are its realization.

References: \citeonline{Bhowmik2020}, \citeonline{Kale2018} 

\section{Cloud Architectures}
\acresetall

There are three cloud computing approaches: SaaS (Software as a Service), Infrastructure as a service (IaaS) and Platform as a service (PaaS). This work is a Software.
The approach used in this work follows the architecture of a SaaS system. SaaS is a service model utilized in Cloud Computing, where the users don’t need physical hardware, nor buy extra software or install, maintain or update software. This service model improved businesses in terms of cost and time efficiency. The Google suite (Gmail, Google Docs, Google Sheets, Google Slides) is currently the prime example of SaaS, as it provides a cloud replacement to popular office software suites such as the Microsoft Office. 
Another striking feature from SaaS is that it requires query processing, where it is performed query propagation and result propagation; The query propagation forwards the query through nodes, where it will then be processed by the main software, then once it finishes it performs the result propagation where the result is obtained by the client program.
As many large applications are typically decomposed into functional primitives nowadays, SaaS proves to be useful, as the cloud services and its APIs are decoupled from major applications and its nature follows isolation and single-responsibility principles; It is especially useful for software utilizing SOA.

References: \citeonline{Kavitha2020} , \citeonline{Singh2019}, \citeonline{Franklin2019}, \citeonline{Banerjee2014}

\subsection{Benefits}
\acresetall

The SaaS approach also offers extra benefits, as when an interoperability with an application occurs, it lessens the use of the main application local server resource usage (instead using the remote server), and adds an extra layer of security as it doesn’t provide direct access to databases, instead it just utilizes smaller samples of queries sent by the main application. Additionally, as it doesn’t require installation, maintenance and updates, it doesn’t require compatibility with software frameworks, operating systems, only requiring the end user to follow a set of required query standards when performing the interoperability. It might improve the QoS (quality of service) of the client application, as it offloads the raw hardware computability requirements, thus improving critical performance and response time, provided that the application is hosted on a server with superior specifications.

References: \citeonline{Garbis2021}

\section{Web Service Architecture}
\acresetall

The overall system architecture will be structured as shown on \ref{fig:client_to_service}.

\begin{figure}[hbt!]
  \centering
  \includesvg[scale=0.25]{client_to_service.svg}
  \caption{The complete Client to Service system architecture for this software}
  \label{fig:client_to_service}
\end{figure}


\subsection{Restful Architeture}
\acresetall

Terei que descrever a estrutura das rotas e os comandos REST aqui

Dar uma melhor definição a Microservices


\subsection{Microservice Architecture}
\acresetall

The software will follow a Microservices architecture approach, it won’t feature persistent database storage shared with the client and the software itself will be completely isolated from the client application, both sides won’t interfere with its usage as long as the service contract isn’t broken. It will operate by receiving JSON queries (where it could stream database information) and return a single JSON query to the client application containing the target recommendation list.

References: \citeonline{saasmicro2015},\citeonline{saasmicrocrucial}, \citeonline{Song2019}


Diferenciar Microservice de SOA

Accordingly to \citeonline{Technologies2019}: "SOA is a manufacturing model which deals with designing and building software by applying the service oriented computing principles to software solutions, while SaaS is a model for sales and distribution of software applications. In simpler terms, SaaS is a means of delivering software as services over the internet to its subscribers, while SOA is an architectural model in which the smallest unit of logic is a service. So, SOA (an architectural strategy) and SaaS (a business model) cannot be directly compared. However, to get the maximum benefits of cost reduction and agility, it is highly recommended that enterprises integrate SOA and SaaS together."


\section{Recommendation as a Service}


Recommendation as a service is mostly an untapped potential for the industry. Many sites, services and platforms need to provide content personalization, at the same time, for both costs and time constraints, it is not feasible to implement a Recommender System implementation of their own. So a system that performs Recommendation as a Service might be the solution. There are commercial and free options available for this end.  The table \ref{tab:cf_methods} lists some of these services that attempt to fulfil that goal.  


Tabela ainda precisa ser atualizada para adicionar mais serviços e corrigir definindo se é comercial ou não

\begin{table}[hbt!]
\begin{tabular}{|l|l|l|l|}
\hline
\textbf{RaaS name}                             & \textbf{Focus}     & \textbf{Commercial}             & \textbf{Actively developed}                                    \\ \hline
\textbf{Recombee}                             & Generic             & Yes & Yes \\ \hline
\textbf{Darwin and Goliath}                             & Generic             & Yes & Yes \\ \hline
\textbf{Mr. DLib}                             & Digital Libraries             & Yes & Yes \\ \hline
\textbf{bX}                             & Digital Libraries             & Yes & Yes \\ \hline
\textbf{BibTip}                             & Digital Libraries             & Yes & Yes \\   \hline
\textbf{yuspify}                             & e-Commerce             & Yes & Yes \\   \hline
\textbf{Kea Labs}                             & e-Commerce             & Yes & Yes \\   \hline
\textbf{Sugestio}                             & Generic             & Yes & No \\   \hline
\textbf{WebTunix}                             & Generic             & Yes & Yes \\   \hline
\end{tabular}
\centering
\caption{Métodos, grupos e algoritmos da técnica de filtragem colaborativa.}
\label{tab:cf_methods}
\end{table}

There is no verifiable popularity in the field. This work intents to use a generalist approach within its Recommender System algorithm, in order to be usable by as many different use cases as possible, be it to provide functionality for eCommerce systems (where it could provide recommendations for products), to provide functionality for media indexation platforms (movies, music), or even specific commercial uses, such as tourism (travel/flight recommendations). The intent of this work is to require the least amount of effort as possible from the end user’s system perspective, as it won’t require installation, deployment, updates or maintain once the environment is already set up on a remote server; It will still require the user to follow guidelines, however, on how the queries should be properly standardized in order to make the interoperability work (well-defined service contracts). This recommender service will support both collaborative filtering and content-based filtering. Where the content filtering will be typically based on the user's preferences and interests, and the collaborative filtering will compare the user’s preferences with the ones from other users.

References: \citeonline{Marko2018}, \citeonline{Baldominos2015}, \citeonline{BenShimon2014}, \citeonline{Furusawa2011}, \citeonline{Han2009}, \citeonline{raas2021}, \citeonline{Teruya2020},\citeonline{Meissa2020},\cite{Jain2019},\cite{Garcia2018},\citeonline{Baldominos2015}